\documentclass[a4paper,10pt]{article}

\usepackage[inner=3cm,top=3cm,outer=2cm,bottom=3cm]{geometry}
\usepackage[utf8]{inputenc}
\usepackage[swedish]{babel}

\title{MIDI Sequencer}
\author{Jacob Adlers & Gustaf Pihl}
\date{Oktober 27, 2016}

\begin{document}
\maketitle

\section{Objective and requirements}
Our goal is to turn the UNO32 Chipkit into a MIDI sequencer where we can use a MIDI keyboard to create sequences of MIDI data. We intend to output from the chipkit in MIDI form so that an external device can parse the output into sound. A sequencer is a electronic device for storing sequences of musical notes and rhythms.\\

The main requirements of our project are:
\begin{itemize}
  \item record sequences from MIDI input
  \item output the stored sequence via MIDI
  \item clear the stored sequence
  \item undo last edit
  \item play/pause
\end{itemize}

If there is time for it we will implement the following:
\begin{itemize}
  \item remove specific note in sequence
  \item adjustable BPM (tempo)
  \item use LCDs to display information about the saved sequence
  \item transpose function
  \item multiple sequences stored
\end{itemize}

\section{Solution}


\section{Validation}
We will connect our MIDI output to an external device and record a sequence. If the sequence is played back by the device then our sequencer is working properly.


\end{document}
