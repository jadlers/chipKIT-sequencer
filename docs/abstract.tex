\documentclass[a4paper,10pt]{article}

\usepackage[inner=3cm,top=3cm,outer=2cm,bottom=3cm]{geometry}
\usepackage[utf8]{inputenc}
\usepackage[swedish]{babel}

\title{MIDI Sequencer}
\author{Jacob Adlers (960331-2472) & Gustaf Pihl (861223-7555)}
\date{Oktober 31, 2016}

\begin{document}
\maketitle

\section{Objective and requirements}
Our goal is to turn the UNO32 Chipkit into a MIDI sequencer where we can use a MIDI keyboard to create sequences of MIDI data. We intend to output from the chipkit in MIDI form so that an external device can parse the output into sound. A sequencer is a electronic device for storing sequences of musical notes and rhythms. We want to qualify as an advanced project.\\

The main requirements of our project are:
\begin{itemize}
  \item Record sequences from MIDI input
  \item Output the stored sequence via MIDI
  \item Clear the stored sequence
  \item Undo last edit
  \item Play/pause
  \item Toggle record enable/disable
\end{itemize}

If there is time for it we will implement the following:
\begin{itemize}
  \item Remove specific note in sequence
  \item Adjustable BPM (tempo)
  \item Use LCDs to display information about the saved sequence
  \item Transpose function
  \item Multiple sequences stored
  \item Mutable quantization
\end{itemize}

\section{Solution}
We will use the ChipKIT Uno 32 board along with the Basic I/O shield to develop our sequencer. The buttons and switches on the I/O shield will be used to implement the features listed in our requirements. We will use a matrix (2D-array) to store input MIDI data on the ChipKIT, where the columns represent time intervals and each row represents one pitch. A timer will loop through the columns of the matrix and send the MIDI data stored in each column using UART to external devices.

\section{Validation}
We will connect our MIDI output to an external device and record a sequence. If the sequence is played back by the device then our sequencer is working properly.


\end{document}
